\begin{center}
    \underbar{\small{Breakpoints}} \\
    \begin{tabular}{|p{2.28cm}|p{2.68cm}|}
        \hline
        command                         & 功能 \\
        \hline
        \verb|[break] [b]|              & 在當前這行放中斷點 \\
        \verb|[b fn]|                   & 在函式 fn 的開頭放中斷點 \\
        \verb|[b N]|                    & 在第 N 行放中斷點 \\
        \verb|[clear N] [cl N]|         & 刪除第 N 行的中斷點 \\
        \verb|[delete] [d]|             & 刪除所有中斷點 \\
        \verb|[d N]|                    & 刪除編號為 N 的中斷點 \\
        \verb|[disable] [dis]|          & 使所有中斷點無效 \\
        \verb|[dis N]|                  & 使編號為 N 的中斷點無效 \\
        \verb|[dp a, "%d\n", c]|        & 碰到第 a 行時印出 c \\
        \verb|[enable] [en]|            & 使所有中斷點有效 \\
        \verb|[en N]|                   & 使編號為 N 的中斷點有效 \\
        \verb|[watch x==3]|             & 執行到符合條件時暫停 \\
        \hline
    \end{tabular} \\
    \hspace{\fill} \\
    \hspace{\fill} \\

    \underbar{\small{Data}} \\
    \begin{tabular}{|p{2.28cm}|p{2.68cm}|}
        \hline
        command                         & 功能 \\
        \hline
        \verb|[call fn]|                & 呼叫函式 fn \\
        \verb|[display x] [disp x]|     & 每執行一步都印出 x \\
        \verb|[print var] [p var]|      & 印出 var \\
        \verb|[undisp x]|               & 取消編號為 x 的 disp \\
        \hline
    \end{tabular} \\
    \hspace{\fill} \\
    \hspace{\fill} \\

    \underbar{\small{Files}} \\
    \begin{tabular}{|p{2.28cm}|p{2.68cm}|}
        \hline
        command                         & 功能 \\
        \hline
        \verb|[list] [l]|               & 印出 10 行程式碼  \\
        \verb|[l N]|                    & 印出包含第 N 行的程式碼  \\
        \verb|[l fn]|                   & 印出包含函式 fn 的程式碼 \\
        \verb|[l var]|                  & 印出包含變數 var 的程式碼 \\
        \hline
    \end{tabular} \\
    \hspace{\fill} \\
    \hspace{\fill} \\

    \underbar{\small{Running}} \\
    \begin{tabular}{|p{2.28cm}|p{2.68cm}|}
        \hline
        command                         & 功能 \\
        \hline
        \verb|[continue] [c]|           & 執行到下個中斷點或錯誤 \\
        \verb|[finish] [fin]|           & 執行到跳出堆疊框 \\
        \verb|[kill] [ki]|              & 終止程式 \\
        \verb|[next] [n]|               & 執行下一行(不進入函式) \\
        \verb|[n N]|                    & 執行 [n] 一共 N 次 \\
        \verb|[run] [r]|                & 執行程式 \\
        \verb|[r < file1]|              & 像 [r], 但輸入為 file1 \\
        \verb|[start]|                  & 開始執行,停在第一步 \\
        \verb|[step] [s]|               & 執行下一步(進入函式) \\
        \verb|[s N]|                    & 執行 [s] 一共 N 次 \\
        \verb|[until N] [u N]|          & 執行到第 N 行停下來 \\
        \hline
    \end{tabular} \\
    \hspace{\fill} \\
    \hspace{\fill} \\

    \underbar{\small{Stack}} \\
    \begin{tabular}{|p{2.28cm}|p{2.68cm}|}
        \hline
        command                         & 功能 \\
        \hline
        \verb|[backtrace] [bt]|         & 印出所有堆疊 \\
        \verb|[frame] [f]|              & 印出當前堆疊 \\
        \verb|[return] [ret]|           & 從當前函式 return \\
        \verb|[up]|                     & 印出上一層堆疊 \\
        \hline
    \end{tabular} \\
    \hspace{\fill} \\
    \hspace{\fill} \\

    \underbar{\small{Status}} \\
    \begin{tabular}{|p{2.28cm}|p{2.68cm}|}
        \hline
        command                         & 功能 \\
        \hline
        \verb|[info] [i]|               & 顯示資訊 \\
        \verb|[i b]|                    & 列出所有中斷點資訊 \\
        \verb|[i disp]|                 & 列出所有監看變數資訊 \\
        \verb|[i local] [i lo]|         & 列出所有區域變數資訊 \\
        \verb|[i var]|                  & 列出所有變數 \\
        \hline
    \end{tabular} \\
    \hspace{\fill} \\
    \hspace{\fill} \\

    \underbar{\small{Support}} \\
    \begin{tabular}{|p{2.28cm}|p{2.68cm}|}
        \hline
        command                         & 功能 \\
        \hline
        \verb|[help] [h]|               & 協助 \\
        \verb|[quit] [q]|               & 結束 gdb \\
        \hline
    \end{tabular}
\end{center}