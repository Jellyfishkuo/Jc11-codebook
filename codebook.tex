\documentclass[6pt,oneside]{article}
\setlength{\columnsep}{18pt}                    %兩欄模式的間距
\setlength{\columnseprule}{0pt}                 %兩欄模式間格線粗細

\usepackage{amsthm}                             %定義,例題
\usepackage{amssymb}
\usepackage{fontspec}                           %設定字體
\usepackage{color}
\usepackage[x11names]{xcolor}
\usepackage{listings}                           %顯示code用的
\usepackage{fancyhdr}                           %設定頁首頁尾
\usepackage{graphicx}                           %Graphic
\usepackage{enumerate}
\usepackage{titlesec}
\usepackage{amsmath}
\usepackage[CheckSingle, CJKmath]{xeCJK}
\usepackage{CJKulem}

\usepackage{amsmath, courier, listings, fancyhdr, graphicx}
\topmargin=0pt
\headsep=5pt
\textheight=740pt
\footskip=0pt
\voffset=-50pt
\textwidth=545pt
\marginparsep=0pt
\marginparwidth=0pt
\marginparpush=0pt
\oddsidemargin=0pt
\evensidemargin=0pt
\hoffset=-42pt

% 以下為改變section的行高
\makeatletter
\let\origsection\section
\renewcommand\section{\@ifstar{\starsection}{\nostarsection}}

\newcommand\nostarsection[1]
{\sectionprelude\origsection{#1}\sectionpostlude}

\newcommand\starsection[1]
{\sectionprelude\origsection*{#1}\sectionpostlude}

\newcommand\sectionprelude{%
  \vspace{-20pt} % 改這個數字
}
\newcommand\sectionpostlude{%
  \vspace{-15pt} % 改這個數字
}
\makeatother

% 設定column數
\usepackage{multicol}
\newcommand{\NumberOfColumn}{3}

%\renewcommand\listfigurename{圖目錄}
%\renewcommand\listtablename{表目錄}

%%%%%%%%%%%%%%%%%%%%%%%%%%%%%

\setmainfont[
    AutoFakeSlant,
    BoldItalicFeatures={FakeSlant},
    UprightFont={* Medium},
    BoldFont={* Bold}
]{Inconsolata}
%\setmonofont{Ubuntu Mono}
\setmonofont[
    AutoFakeSlant,
    BoldItalicFeatures={FakeSlant},
    UprightFont={* Medium},
    BoldFont={* Bold}
]{Inconsolata}
\setCJKmainfont[
  SizeFeatures={Size=6}
]{Noto Sans CJK TC}
\XeTeXlinebreaklocale "zh"                      %中文自動換行
\XeTeXlinebreakskip = 0pt plus 1pt              %設定段落之間的距離
\setcounter{secnumdepth}{3}                     %目錄顯示第三層

%%%%%%%%%%%%%%%%%%%%%%%%%%%%%
\makeatletter
\lst@CCPutMacro\lst@ProcessOther {"2D}{\lst@ttfamily{-{}}{-{}}}
\@empty\z@\@empty
\makeatother
\lstset{                                        % Code顯示
    language=C++,                               % the language of the code
    basicstyle=\footnotesize\ttfamily,          % the size of the fonts that are used for the code
    numbers=left,                               % where to put the line-numbers
    numberstyle=\scriptsize,                    % the size of the fonts that are used for the line-numbers
    stepnumber=1,                               % the step between two line-numbers. If it's 1, each line  will be numbered
    numbersep=5pt,                              % how far the line-numbers are from the code
    backgroundcolor=\color{white},              % choose the background color. You must add \usepackage{color}
    showspaces=false,                           % show spaces adding particular underscores
    showstringspaces=false,                     % underline spaces within strings
    showtabs=false,                             % show tabs within strings adding particular underscores
    frame=false,                                % adds a frame around the code
    tabsize=2,                                  % sets default tabsize to 2 spaces
    captionpos=b,                               % sets the caption-position to bottom
    breaklines=true,                            % sets automatic line breaking
    breakatwhitespace=true,                     % sets if automatic breaks should only happen at whitespace
    escapeinside={\%*}{*)},                     % if you want to add a comment within your code
    morekeywords={*},                           % if you want to add more keywords to the set
    keywordstyle=\bfseries\color{Blue1},
    commentstyle=\itshape\color{Red1},
    stringstyle=\itshape\color{Green4},
}


\begin{document}
\pagestyle{fancy}
\fancyfoot{}
%\fancyfoot[R]{\includegraphics[width=20pt]{ironwood.jpg}}
\fancyhead[C]{FJCU}
\fancyhead[L]{Jc11}
\fancyhead[R]{\thepage}
\renewcommand{\headrulewidth}{0.4pt}
\renewcommand{\contentsname}{Contents}

\scriptsize
\tableofcontents

\section{Ubuntu}
%     \subsection{run}
%         \lstinputlisting{ubuntu/run.tex}
%     \subsection{cp.sh}
%         \lstinputlisting{ubuntu/cp.sh}
    \subsection{Terminal Title}
        方法:
\begin{lstlisting}
    PS1='\e];\a'
\end{lstlisting}
舉例:
\begin{lstlisting}
    PS1='\e];\W\a\w$ '
    // 可改 \W 為想要的 Title
\end{lstlisting}
\begin{itemize}
    \item \verb|[\a]| - ASCII Bell
    \item \verb|[\d]| - 日期
    \item \verb|[\e]| - 跳脫字元
    \item \verb|[\h]| - 主機
    \item \verb|[\H]| - 主機名
    \item \verb|[\t]| - 時間  
    \item \verb|[\u]| - 使用者
    \item \verb|[\w]| - 當前路徑
    \item \verb|[\W]| - 當前資料夾名稱
\end{itemize}
    \subsection{GDB 參數}
        \begin{lstlisting}
    g++ main.cpp -g -o main
    gdb -tui -q ./main
\end{lstlisting}
\verb|[-tui]| - 在終端機顯示文字檔案 \\
\verb|[-q]  | - 在初始設定不顯示版本資訊 \\
    \subsection{GDB 指令}
        \begin{center}
    \underbar{\small{Breakpoints}} \\
    \begin{tabular}{|p{2.7cm}|p{2.7cm}|}
        \hline
        command                         & 功能 \\
        \hline
        \verb|[break] [b]|              & 在當前這行放中斷點 \\
        \verb|[b fn]|                   & 在函式 fn 的開頭放中斷點 \\
        \verb|[b N]|                    & 在第 N 行放中斷點 \\
        \verb|[clear N] [cl N]|         & 刪除第 N 行的中斷點 \\
        \verb|[command N] [comm N]|     & 設定編號 N 的中斷點的指令 \\
        \verb|[cond N i==3]|            & 編號 N 的中斷點i=3再停 \\
        \verb|[delete] [d]|             & 刪除所有中斷點 \\
        \verb|[d N]|                    & 刪除編號為 N 的中斷點 \\
        \verb|[disable] [dis]|          & 使所有中斷點無效 \\
        \verb|[dis N]|                  & 使編號為 N 的中斷點無效 \\
        \verb|[dp a, "%d\n", c]|        & 碰到第 a 行時印出 c \\
        \verb|[enable] [en]|            & 使所有中斷點有效 \\
        \verb|[en N]|                   & 使編號為 N 的中斷點有效 \\
        \verb|[tbreak N] [tb N]|        & 只停一次的 [b N] \\
        \verb|[watch x==3] [wa x==3]|   & 執行到符合條件時停止 \\
        \hline
    \end{tabular} \\
    \hspace{\fill} \\
    \hspace{\fill} \\

    \underbar{\small{Data}} \\
    \begin{tabular}{|p{2.7cm}|p{2.7cm}|}
        \hline
        command                         & 功能 \\
        \hline
        \verb|[call fn]|                & 呼叫函式 fn \\
        \verb|[display x] [disp x]|     & 每執行一步都印出 x \\
        \verb|[print var] [p var]|      & 印出 var \\
        \verb|[set print] [set p]|      & 設定 print \\
        \verb|[set p array]|            & array 印出漂亮格式 \\
        \verb|[set p array off]|        & 取消 array 印出漂亮格式 \\
        \verb|[set p array-i]|          & array 印出索引 \\
        \verb|[set p array-i off]|      & 取消 array 印出索引 \\
        \verb|[set p el N]|             & array 最多印 N 個元素 \\
        \verb|[set p pre]|              & struct 印出漂亮格式 \\
        \verb|[set p pre off]|          & 取消 struct 印出漂亮格式 \\
        \verb|[set var N=3]|            & 將 N 設為 3 \\
        \verb|[undisp x]|               & 取消編號為 x 的 disp \\
        \hline
    \end{tabular}
    \columnbreak

    \underbar{\small{Files}} \\
    \begin{tabular}{|p{2.7cm}|p{2.7cm}|}
        \hline
        command                         & 功能 \\
        \hline
        \verb|[list] [l]|               & 印出 10 行程式碼  \\
        \verb|[l N]|                    & 印出包含第 N 行的程式碼  \\
        \verb|[l fn]|                   & 印出包含函式 fn 的程式碼 \\
        \verb|[l var]|                  & 印出包含變數 var 的程式碼 \\
        \hline
    \end{tabular} \\
    \hspace{\fill} \\
    \hspace{\fill} \\

    \underbar{\small{Obscure}} \\
    \begin{tabular}{|p{2.7cm}|p{2.7cm}|}
        \hline
        command                         & 功能 \\
        \hline
        \verb|[record] [rec]|           & 開始記錄 \\
        \verb|[rec s]|                  & 停止記錄 \\
        \hline
    \end{tabular} \\
    \hspace{\fill} \\
    \hspace{\fill} \\

    \underbar{\small{Running}} \\
    \begin{tabular}{|p{2.7cm}|p{2.7cm}|}
        \hline
        command                         & 功能 \\
        \hline
        \verb|[continue] [c]|           & 執行到下個中斷點或錯誤 \\
        \verb|[finish] [fin]|           & 執行到跳出堆疊框 \\
        \verb|[kill] [k]|               & 終止程式 \\
        \verb|[next] [n]|               & 執行下一行(不進入函式) \\
        \verb|[n N]|                    & 執行 [n] 一共 N 次 \\
        \verb|[reverse-continue] [rc]|  & 反向的 [c] \\
        \verb|[rn]|                     & 反向的 [n] \\
        \verb|[rs]|                     & 反向的 [s] \\
        \verb|[run] [r]|                & 執行程式 \\
        \verb|[r < file1]|              & 像 [r], 但輸入為 file1 \\
        \verb|[start]|                  & 開始執行,停在第一步 \\
        \verb|[step] [s]|               & 執行下一步(進入函式) \\
        \verb|[s N]|                    & 執行 [s] 一共 N 次 \\
        \verb|[until N] [u N]|          & 執行到第 N 行停下來 \\
        \hline
    \end{tabular} \\
    \hspace{\fill} \\
    \hspace{\fill} \\

    \underbar{\small{Stack}} \\
    \begin{tabular}{|p{2.7cm}|p{2.7cm}|}
        \hline
        command                         & 功能 \\
        \hline
        \verb|[backtrace] [bt]|         & 印出所有堆疊 \\
        \verb|[down] [do]|              & 印出下一層堆疊 \\
        \verb|[frame] [f]|              & 印出當前堆疊 \\
        \verb|[f N]|                    & 印出往上第 N 層堆疊 \\
        \verb|[return] [ret]|           & 從當前函式 return \\
        \verb|[up]|                     & 印出上一層堆疊 \\
        \hline
    \end{tabular} \\
    \hspace{\fill} \\
    \hspace{\fill} \\

    \underbar{\small{Status}} \\
    \begin{tabular}{|p{2.7cm}|p{2.7cm}|}
        \hline
        command                         & 功能 \\
        \hline
        \verb|[info] [i]|               & 顯示資訊 \\
        \verb|[i b]|                    & 列出所有中斷點資訊 \\
        \verb|[i disp]|                 & 列出所有監看變數資訊 \\
        \verb|[i local] [i lo]|         & 列出所有區域變數資訊 \\
        \verb|[i var]|                  & 列出所有變數 \\
        \hline
    \end{tabular} \\
    \hspace{\fill} \\
    \hspace{\fill} \\

    \underbar{\small{Support}} \\
    \begin{tabular}{|p{2.7cm}|p{2.7cm}|}
        \hline
        command                         & 功能 \\
        \hline
        \verb|[help] [h]|               & 協助 \\
        \verb|[b N if i==2]|            & 當 i=2 時停在第 N 行 \\
        \verb|[quit] [q]|               & 結束 gdb \\
        \hline
    \end{tabular} \\
    \hspace{\fill} \\
    \hspace{\fill} \\

    \underbar{\small{Text User Interface}} \\
    \begin{tabular}{|p{2.7cm}|p{2.7cm}|}
        \hline
        command                         & 功能 \\
        \hline
        \verb|[refresh] [ref]|          & 刷新終端機布置 \\
        \verb|[tui d]|                  & 取消使用 TUI \\
        \verb|[tui e]|                  & 使用 TUI \\
        \verb|[update] [upd]|           & 更新視窗以顯示當前程式碼 \\
        \hline
    \end{tabular} \\
    \hspace{\fill} \\
    \hspace{\fill} \\

    \underbar{\small{Others}} \\
    \begin{itemize}
        \item 按 <enter> 鍵可以執行上一動
        \item 執行 reverse (像是 rc, rn, rs) 前,\\
              要先執行 record (rec),但存不了幾步就沒空間了
        \item 輸入命令 command (comm) 後,接下來每一行輸入一個命令,以 end 作結,之後執行到這裡都會執行所有命令。 \\
              同一個中斷點有變動其 command 時會完全按照新的輸入。
    \end{itemize}
\end{center}
\clearpage

\section{字串}
    \subsection{最長迴文子字串}
        \lstinputlisting{字串/最長迴文子字串.cpp} 
    \subsection{Manacher}
        \begin{description}
    \item [s:] 增長為兩倍的字串,以\verb|'@'|為首,以\verb|'$'|為間隔,以\verb|'\0'|節尾
    \item [p:] 以 s[i] 為中心,半徑為 p[i] 是迴文
    \item [return:] 最長的迴文長度
\end{description}
\begin{lstlisting}
const int maxn = 1e5 + 10;

char s[maxn<<1] = "@$";
int p[maxn<<1];

int manacher(char* str, int n) {
  for(int i=1; i<=n; i++) {
    s[i<<1] = str[i-1];
    s[i<<1|1] = '$';
  }

  int cur = 0, r = 0, res = 0;
  s[n = (n+1) << 1] = 0;
  for(int i=1; i<n; i++) {
    p[i] = (i>r) ? 1 : min(p[cur*2-i], r-i);
    for(; s[i-p[i]]==s[i+p[i]]; p[i]++);
    if(i+p[i] > r) {
      r = i + p[i];
      cur = i;
    }
    res = max(res, p[i]);
  }
  return res - 1;
}
\end{lstlisting} \columnbreak
    \subsection{KMP}
        \lstinputlisting{字串/KMP.cpp}
    \subsection{Z Algorithm}
        \lstinputlisting{字串/Z_function.cpp}  \columnbreak
    \subsection{Suffix Array}
        \begin{itemize}
  \item $O(n\log(n))$
  \item SA:後綴數組
  \item HA:相鄰後綴的共同前綴長度 \par
        (Longest Common Prefix)
  \item maxc:可用字元的最大ASCII值
  \item maxn >= maxc
  \item 記得先取 n 的值(strlen(s))
\end{itemize}
\begin{lstlisting}
const int maxn = 2e5 + 10;
const int maxc = 256 + 10;

int n;
int SA[maxn], HA[maxn];
int rk[maxn], cnt[maxn], tmp[maxn];
char s[maxn];

void getSA() {
  int mx = maxc;
  for(int i=0; i<mx; cnt[i++]=0);

  // 第一次 stable counting sort,編 rank 和 sa
  for(int i=0; i<n; i++) cnt[rk[i]=s[i]]++;
  for(int i=1; i<mx; i++) cnt[i] += cnt[i-1];
  for(int i=n-1;i>=0;i--) SA[--cnt[s[i]]]=i;

  // 倍增法運算
  for(int k=1, r=0; k<n; k<<=1, r=0) {
    for(int i=0; i<mx; cnt[i++]=0);
    for(int i=0; i<n; i++) cnt[rk[i]]++;
    for(int i=1; i<mx; i++) cnt[i]+=cnt[i-1];
    for(int i=n-k; i<n; i++) tmp[r++] = i;
    for(int i=0; i<n; i++) {
      if(SA[i] >= k) tmp[r++] = SA[i] - k;
    }

    // 計算本回 SA
    for(int i=n-1; i>=0; i--) {
      SA[--cnt[rk[tmp[i]]]] = tmp[i];
    }

    // 計算本回 rank
    tmp[SA[0]] = r = 0;
    for(int i=1; i<n; i++) {
      if((SA[i-1]+k >= n) ||
         (rk[SA[i-1]] != rk[SA[i]]) ||
         (rk[SA[i-1]+k] != rk[SA[i]+k])) r++;
      tmp[SA[i]] = r;
    }
    for(int i=0; i<n; i++) rk[i] = tmp[i];
    if((mx=r+1) == n) break;
  }
}

void getHA() {    // HA[0] = 0
  for(int i=0; i<n; i++) rk[SA[i]] = i;
  for(int i=0, k=0; i<n; i++) {
    if(!rk[i]) continue;
    if(k) k--;
    while(s[i+k] == s[SA[rk[i]-1]+k]) k++;
    HA[rk[i]] = k;
  }
}
\end{lstlisting}
\clearpage

\section{math}
    \subsection{公式}
        \begin{itemize}
    \item Faulhaber's formula
\end{itemize}
\[ \sum_{k=1}^{n} k^p = \frac{1}{p+1} \sum_{r=0}^{p} \binom{p+1}{r} B_rn^{p-r+1} \]
\[ \text{where} \ B_0 = 1, \ B_r = 1-\sum_{i=0}^{r-1} \binom{r}{i} \frac{B_i}{r-i+1} \]
\[ 也可用高斯消去法找deg(p+1)的多項式,例: \]
\[ \sum_{k=1}^{n} k^2 = a_3n^3 + a_2n^2 + a_1n + a_0 \]
\tiny\[
    \begin{bmatrix}
        0^3 & 0^2 & 0^1 & 0^0\\
        1^3 & 1^2 & 1^1 & 1^0\\
        2^3 & 2^2 & 2^1 & 2^0\\
        3^3 & 3^2 & 3^1 & 3^0
    \end{bmatrix}
    \begin{bmatrix}
        a_3 \\ a_2 \\ a_1 \\ a_0
    \end{bmatrix} =
    \begin{bmatrix}
        0^2 \\ 0^2+1^2 \\ 0^2+1^2+2^2 \\ 0^2+1^2+2^2+3^2
    \end{bmatrix}
\]
\tiny\[
    \begin{bmatrix}
        0 & 0 & 0 & 1 & 0 \\
        1 & 1 & 1 & 1 & 1 \\
        8 & 4 & 2 & 1 & 5 \\
        27 & 9 & 3 & 1 & 14
    \end{bmatrix}
  \Rightarrow \begin{bmatrix}
        1 & 1 & 1 & 1 & 1 \\
        0 & 4 & 6 & 7 & 3 \\
        0 & 0 & 6 & 11 & 1 \\
        0 & 0 & 0 & 1 & 0
    \end{bmatrix}, 
    A = \begin{bmatrix}
        1/3 \\ 1/2 \\ 1/6 \\ 0 
    \end{bmatrix}
\]
\[ \sum_{k=1}^{n} k^2 = \frac{1}{3}n^3 + \frac{1}{2}n^2 + \frac{1}{6}n \]
    \subsection{Rational}
        \lstinputlisting{math/Rational.cpp}
    \subsection{乘法逆元、組合數}
        \[
    \begin{array}{l}
        x^{-1} mod \ m \\
        \tiny{
            = \begin{cases}
                \begin{array}{cl}
                        1, & \text{if} \ x=1 \\
                        -\left \lfloor\frac{m}{x} \right\rfloor(m \ mod \ x)^{-1}, & \text{otherwise}
                \end{array}
            \end{cases}(mod \ m)
        } \\
        \tiny{
            = \begin{cases}
                \begin{array}{cl}
                        1, & \text{if} \ x=1 \\
                        (m-\left \lfloor\frac{m}{x} \right\rfloor)(m \ mod \ x)^{-1}, & \text{otherwise}
                \end{array}
            \end{cases}(mod \ m)
        }
    \end{array}
\]

{\raggedright
    \(若 \ p \in prime, 根據費馬小定理, 則 \) \par
    \(
        \begin{array}{rrcl}
                \because & ax & \equiv & 1 \ (mod \ p) \\
                \therefore & ax & \equiv & a^{p-1} \ (mod \ p) \\
                \therefore & x & \equiv & a^{p-2} \ (mod \ p)
        \end{array}
    \) \par
}
    \subsection{歐拉函數}
        \lstinputlisting{math/歐拉函數.cpp} \columnbreak
    \subsection{質數與因數}
        \lstinputlisting{math/質數與因數.cpp}
    \subsection{Pisano Period}
        \lstinputlisting{math/PisanoPeriod.cpp}
    \subsection{矩陣快速冪}
        \lstinputlisting{math/矩陣快速冪.cpp}
    % \subsection{快速冪}
    %     \lstinputlisting{math/快速冪.cpp} 
    % \subsection{atan}
    %     \lstinputlisting{math/atan.cpp}
    \subsection{大步小步}
        \lstinputlisting{math/大步小步.cpp} \columnbreak
    \subsection{高斯消去}
        \begin{description}
    \item 計算 $AX = B$
    \item 傳入: \\
        $M=$ 增廣矩陣 $[A|B]$ \\
        $equ=$ 有幾個 equation \\
        $var=$ 有幾個 variable
    \item 回傳:$X = (x_0,\dots, x_{n-1})$ 的解集
    \item >>無法判斷無解或無限多組解<<
\end{description}

\begin{lstlisting}
using DBL = double;
using mat = vector<vector<DBL>>;

vector<DBL> Gauss(mat& M, int equ, int var) {
  auto dcmp = [](DBL a, DBL b=0.0) {
    return (a > b) - (a < b);
  };

  for(int r=0, c=0; r<equ && c<var; ) {
    int mx = r;   // 找絕對值最大的 M[i][c]
    for(int i=r+1; i<equ; i++) {
      if(dcmp(abs(M[i][c]),abs(M[mx][c]))==1)
        mx = i;
    }
    if(mx != r) swap(M[mx], M[r]);

    if(dcmp(M[r][c]) == 0) {
      c++;
      continue;
    }

    for(int i=r+1; i<equ; i++) {
      if(dcmp(M[i][c]) == 0) continue;
      DBL t = M[i][c] / M[r][c];
      for(int j=c; j<M[c].size(); j++) {
        M[i][j] -= t * M[r][j];
      }
    }
    r++, c++;
  }

  vector<DBL> X(var);
  for(int i=var-1; i>=0; i--) {
    X[i] = M[i][var];
    for(int j=var-1; j>i; j--) {
      X[i] -= M[i][j] * X[j];
    }
    X[i] /= M[i][i];
  }
  return X;
}
\end{lstlisting}
\clearpage

% \section{STL}
%     % \subsection{deque}
%     %     \lstinputlisting{STL/deque.cpp}
%     % \subsection{map}
%     %     \lstinputlisting{STL/map.cpp}
%     % \subsection{unordered\_map}
%     %     \lstinputlisting{STL/unorderedmap.cpp}
%     % \subsection{set}
%     %     \lstinputlisting{STL/set.cpp}
%     \subsection{multiset}
%         \lstinputlisting{STL/multiset.cpp}
%     \subsection{unordered\_set}
%         \lstinputlisting{STL/unorderedset.cpp}
    
% \section{sort}
%     \subsection{大數排序}
%         \lstinputlisting{sort/bignumbersort.py}

        
\section{algorithm}
    % \subsection{basic}
    %     \lstinputlisting{algorithm/basic.cpp}
    \subsection{greedy}
        \lstinputlisting{algorithm/greedy.cpp} \columnbreak
    
    % \subsection{basic}
    %     \lstinputlisting{algorithm/basic.cpp}
    \subsection{二分搜}
        \lstinputlisting{algorithm/二分搜.cpp}
    \subsection{三分搜}
        \lstinputlisting{algorithm/三分搜.cpp} \columnbreak
    % \subsection{prefix sum}
    %     \lstinputlisting{algorithm/prefixsum.cpp}
    \subsection{dinic}
        \lstinputlisting{algorithm/dinic.cpp} \columnbreak
    % \subsection{Nim Game}
    %     \lstinputlisting{algorithm/NimGame.cpp}
    \subsection{SPFA}
        \lstinputlisting{algorithm/SPFA.cpp}
    \subsection{dijkstra}
        \lstinputlisting{algorithm/dijkstra.cpp}
    \subsection{JosephusProblem}
        \lstinputlisting{algorithm/JosephusProblem.cpp}
    \subsection{SCC Kosaraju}
        \lstinputlisting{algorithm/SCC_Kosaraju.cpp} \columnbreak
    \subsection{SCC Tarjan}
        \lstinputlisting{algorithm/SCC_Tarjan.cpp} \columnbreak
    \subsection{ArticulationPoints Tarjan}
        \lstinputlisting{algorithm/ArticulationPoints_Tarjan.cpp} \columnbreak
    \subsection{最小樹狀圖}
        \lstinputlisting{algorithm/最小樹狀圖_無註解.cpp} \columnbreak
    \subsection{KM}
        \lstinputlisting{algorithm/KM_無註解.cpp}
    \subsection{二分圖最大匹配}
        \lstinputlisting{algorithm/二分圖最大匹配.cpp}
    \subsection{莫隊}
        \lstinputlisting{algorithm/Mo_s_algorithm.cpp} \columnbreak
    \subsection{Blossom Algorithm}
        \lstinputlisting{algorithm/blossom.cpp}
    \subsection{Dancing Links}
        \lstinputlisting{algorithm/DLX.cpp} \columnbreak
    \subsection{Astar}
        \lstinputlisting{algorithm/Astar.cpp}
    \subsection{差分}
        \lstinputlisting{algorithm/差分.cpp} \columnbreak
    \subsection{MCMF}
        \lstinputlisting{algorithm/MCMF_無註解.cpp} \columnbreak
    \subsection{LCA倍增法}
        \lstinputlisting{algorithm/LCA倍增法.cpp} \columnbreak
    \subsection{LCA樹壓平RMQ}
        \lstinputlisting{algorithm/LCA樹壓平RMQ.cpp}
    \subsection{LCA樹鍊剖分}
        \lstinputlisting{algorithm/LCA樹鍊剖分.cpp}
\clearpage

\section{DataStructure}
    \subsection{BIT}
        \lstinputlisting{DataStructure/BIT.cpp}
    \subsection{帶權併查集}
        \begin{itemize}[leftmargin=2em, listparindent=-2em]
    \item val[x] 為 x 到 p[x] 的距離(隨題目變化更改)
    \item merge(u, v, w)
        \subitem $u \stackrel{w}{\longrightarrow} v$
        \subitem $pu=pv$ 時,$val[v]-val[u] \ne w$ 代表有誤
    \item 若 $[l, r]$的總和為$w$,則應呼叫 merge(l-1, r, w)
\end{itemize} \columnbreak
    \subsection{Trie}
        \lstinputlisting{DataStructure/Trie.cpp} \columnbreak
    \subsection{AC Trie}
        \lstinputlisting{DataStructure/AC_Trie.cpp} \columnbreak
    \subsection{線段樹1D}
        \lstinputlisting{DataStructure/SegmentTree1D.cpp} \columnbreak
    \subsection{線段樹2D}
        \lstinputlisting{DataStructure/SegmentTree2D.cpp} \columnbreak
    \subsection{權值線段樹}
        \lstinputlisting{DataStructure/WeightSegmentTree.cpp} \columnbreak
    \subsection{ChthollyTree}
        \lstinputlisting{DataStructure/ChthollyTree.cpp} 
    \subsection{單調隊列}
        \lstinputlisting{DataStructure/monotonicqueue.cpp}
\clearpage
        
\section{Geometry}
    \subsection{公式}
        % 新增格式為:
% \normalsize \item "標題" \par
%     \tiny \input{"檔案在Jc11-codebook底下的完整路徑"}

% Notice:
% 因為中文會造成字型縮小,可以的話"標題"請盡量用英文

\begin{enumerate}[leftmargin=3em]
    \normalsize \item Circle and Line \par
        \small 點$P(x_0, y_0)$ \par
到直線$L:ax+by+c=0$的距離 
\[
    d(P, L)=\frac{|ax_0+by_0+c|}{\sqrt{a^2+b^2}}
\]
兩平行直線$L_1:ax+by+c_1=0$ \par
與$L_2:ax+by+c_2=0$的距離 \par
\[
    d(L_1, L_2) = \frac{|c_1-c_2|}{\sqrt{a^2+b^2}}
\]
    \normalsize \item Triangle \par
        \small 設三角形頂點為$A(x_1, y_1), B(x_2, y_2), C(x_3, y_3)$ \par
點$A, B, C$的對邊長分別為$a, b, c$ \par
三角形面積為$\Delta$ \par
重心為$(G_x, G_y)$,內心為$(I_x, I_y)$, \par
外心為$(O_x, O_y)$和垂心為$(H_x, H_y)$ \par
\[
    \Delta = \frac{1}{2}
    \begin{vmatrix}
        x_1 & y_1 & 1\\
        x_2 & y_2 & 1\\
        x_3 & y_3 & 1
    \end{vmatrix}
\]
\[
    G_x = \frac{1}{3}\left(x_1+x_2+x_3\right)
\]
\[
    G_y = \frac{1}{3}\left(y_1+y_2+y_3\right)
\]
\[
    I_x = \frac{ax_1+bx_2+cx_3}{a+b+c}
\]
\[
    I_y = \frac{ay_1+by_2+cy_3}{a+b+c}
\]
\[
    O_x = \frac{1}{4\Delta}
        \begin{vmatrix}
            x_1^2+y_1^2 & y_1 & 1\\
            x_2^2+y_2^2 & y_2 & 1\\
            x_3^2+y_3^2 & y_3 & 1
        \end{vmatrix}
\]
\[
    O_y = \frac{1}{4\Delta}
        \begin{vmatrix}
            x_1 & x_1^2+y_1^2 & 1\\
            x_2 & x_2^2+y_2^2 & 1\\
            x_3 & x_3^2+y_3^2 & 1
        \end{vmatrix}
\]
\[
    H_x = -\frac{1}{2\Delta}
        \begin{vmatrix}
            x_2x_3+y_2y_3 & y_1 & 1\\
            x_1x_3+y_1y_3 & y_2 & 1\\
            x_1x_2+y_1y_2 & y_3 & 1
        \end{vmatrix}
\]
\[
    H_y = -\frac{1}{2\Delta}
        \begin{vmatrix}
            x_1 & x_2x_3+y_2y_3 & 1\\
            x_2 & x_1x_3+y_1y_3 & 1\\
            x_3 & x_1x_2+y_1y_2 & 1
        \end{vmatrix}
\]

任意三角形,重心、外心、垂心共線 \par
\[
      G_x = \frac{2}{3}O_x+\frac{1}{3}H_x
\]
\[
      G_y = \frac{2}{3}O_y+\frac{1}{3}H_y
\]
\end{enumerate}
    \subsection{Template}
        \begin{center}
    \underbar{\normalsize{Predefined Variables}} \\
    \lstinputlisting{Geometry/Template/predefined variables.cpp}
    \hspace{\fill} \\
    \hspace{\fill} \\

    \underbar{\normalsize{Vector、Point}} \\
    \lstinputlisting{Geometry/Template/Vector-Point.cpp}
    \hspace{\fill} \\
    \hspace{\fill} \\

    \underbar{\normalsize{Line}} \\
    \lstinputlisting{Geometry/Template/Line.cpp}
    \columnbreak

    \underbar{\normalsize{Segment}} \\
    \lstinputlisting{Geometry/Template/Segment.cpp}
    \hspace{\fill} \\
    \hspace{\fill} \\

    \underbar{\normalsize{Circle}} \\
    \lstinputlisting{Geometry/Template/Circle.cpp}
    \hspace{\fill} \\
    \hspace{\fill} \\
\end{center}
    \subsection{旋轉卡尺}
        \lstinputlisting{Geometry/rotating_calipers.cpp}
    \subsection{半平面相交}
        \begin{center}
\underbar{Template}
\begin{lstlisting}
using DBL = double;
using Tp = DBL;                 // 存點的型態
const int maxn = 5e4 + 10;
const DBL eps = 1e-9;
struct Vector;
using Point = Vector;
using Polygon = vector<Point>;
Vector operator+(Vector, Vector);
Vector operator-(Vector, Vector);
Vector operator*(Vector, DBL);
Tp cross(Vector, Vector);
struct Line;
Point intersection(Line, Line);
int dcmp(DBL, DBL);             // 不見得會用到
\end{lstlisting}
\underbar{Halfplane Intersection}
\begin{lstlisting}
// Return: 能形成半平面交的凸包邊界點
Polygon halfplaneIntersect(vector<Line>&nar){
  sort(nar.begin(), nar.end());
  // p 是否在 l 的左半平面
  auto lft = [&](Point p, Line l) {
    return dcmp(cross(l.v, p-l.p)) > 0;
  };

  int ql = 0, qr = 0;
  Line L[maxn] = {nar[0]};
  Point P[maxn]; 

  for(int i=1; i<nar.size(); i++) {
    for(; ql<qr&&!lft(P[qr-1],nar[i]); qr--);
    for(; ql<qr&&!lft(P[ql],nar[i]); ql++);
    L[++qr] = nar[i];
    if(dcmp(cross(L[qr].v,L[qr-1].v))==0) {
      if(lft(nar[i].p,L[--qr])) L[qr]=nar[i];
    }
    if(ql < qr)
      P[qr-1] = intersection(L[qr-1], L[qr]);
  }
  for(; ql<qr && !lft(P[qr-1], L[ql]); qr--);
  if(qr-ql <= 1) return {};
  P[qr] = intersection(L[qr], L[ql]);
  return Polygon(P+ql, P+qr+1);
}
\end{lstlisting}
\end{center}
    \subsection{Polygon}
        \lstinputlisting{Geometry/polygon.cpp} \columnbreak
        \subsection{凸包}
        \begin{itemize}
    \item Tp 為 Point 裡 x 和 y 的型態
    \item struct Point 需要加入並另外計算的 variables: \\
        1. ang, 該點與基準點的atan2值 \\
        2. d2, 該點與基準點的$(距離)^2$
    \item 注意計算 d2 的型態範圍限制
\end{itemize}

\begin{center}
\underbar{Template}
\begin{lstlisting}
using DBL = double;
using Tp = long long;             // 存點的型態
const DBL eps = 1e-9;
const Tp inf = 1e9;               // 座標極大值
struct Vector;
using Point = Vector;
using Polygon = vector<Point>;
Vector operator-(Vector, Vector);
Tp cross(Vector, Vector);
int dcmp(DBL, DBL);
\end{lstlisting}
\underbar{Convex Hull}
\begin{lstlisting}
Polygon convex_hull(Point* p, int n) {
  auto rmv = [](Point a, Point b, Point c) {
    return cross(b-a, c-b) <= 0;  // 非浮點數
    return dcmp(cross(b-a, c-b)) <= 0;
  };

  // 選最下裡最左的當基準點,可在輸入時計算
  Tp lx = inf, ly = inf;
  for(int i=0; i<n; i++) {
    if(p[i].y<ly || (p[i].y==ly&&p[i].x<lx)){
      lx = p[i].x, ly = p[i].y;
    } 
  }

  for(int i=0; i<n; i++) {
    p[i].ang=atan2(p[i].y-ly,p[i].x-lx);
    p[i].d2 = (p[i].x-lx)*(p[i].x-lx) +
              (p[i].y-ly)*(p[i].y-ly);
  }
  sort(p, p+n, [&](Point& a, Point& b) {
    if(dcmp(a.ang, b.ang))
      return a.ang < b.ang;
    return a.d2 < b.d2;
  });

  int m = 1;     // stack size
  Point st[n] = {p[n] = p[0]};
  for(int i=1; i<=n; i++) {
    for(;m>1&&rmv(st[m-2],st[m-1],p[i]);m--);
    st[m++] = p[i];
  }
  return Polygon(st, st+m-1);
}
\end{lstlisting}
\end{center}
        \subsection{最小圓覆蓋}
        \lstinputlisting{Geometry/MEC.cpp}
        \subsection{交點、距離}
            \lstinputlisting{Geometry/intersection-distance.cpp}
\clearpage

\section{DP}
    \subsection{背包}
        \begin{center}
  \underline{  \bf\centering{0-1 背包}  }
\end{center}
\begin{description}
  \item[複雜度:] $O(NW)$
  \item[已知:] 第$i$個物品重量為$w_i$,價值$v_i$;背包總容量$W$
  \item[意義:] dp[前i個物品][重量] = 最高價值
  \item[maxn:] 物品數量
  \item[maxw:] 背包最大容量
\end{description}
\begin{lstlisting}
int W;
int w[maxn], v[maxn];
int dp[maxw];

memset(dp, 0, sizeof(dp));
for(int i=1; i<=n; i++) {
  for(int j=W; j>=w[i]; j--) {
    dp[j] = max(dp[j], dp[j-w[i]]+v[i]);
  }
}
\end{lstlisting}

\begin{center}
  \underline{  \bf\centering{價值為主的 0-1 背包}  }
\end{center}
\begin{description}
  \item[複雜度:] $O(NV)$
  \item[已知:] 第$i$個物品重量為$w_i$,價值$v_i$;物品最大總價值$V$
  \item[意義:] dp[前i個物品][價值] = 最小重量
  \item[maxn:] 物品數量
  \item[maxv:] 物品最大總價值
  \item[$V=\Sigma v_i$] 
\end{description}
\begin{lstlisting}
int w[maxn], v[maxn];
int dp[maxv];

memset(dp, 0x3f, sizeof(dp));
dp[0] = 0;
for(int i=0; i<n; i++) {
  for(int j=V; j>=v[i]; j--) {
    dp[j] = min(dp[j], dp[j-v[i]]+w[i]);
  }
}

int res = 0;
for(int val=V; val>=0; val--) {
  if(dp[val] <= w) {
    res = val;
    break;
  }
}
\end{lstlisting}

\begin{center}
  \underline{  \bf\centering{完全背包(無限背包)}  }
\end{center}
\begin{description}
  \item[複雜度:] $O(NW)$
  \item[已知:] 第$i$個物品重量為$w_i$,價值$v_i$;背包總容量$W$
  \item[意義:] dp[前i個物品][重量] = 最高價值
  \item[maxn:] 物品數量
  \item[maxw:] 背包最大容量
\end{description}
\begin{lstlisting}
int W;
int w[maxn], v[maxn];
int dp[maxw];

memset(dp, 0, sizeof(dp));
for(int i=1; i<=n; i++)
  for(int j=w[i]; j<=W; j++)
    dp[j] = max(dp[j], dp[j-w[i]]+v[i]);
\end{lstlisting}

\begin{center}
  \underline{  \bf\centering{多重背包}  }
\end{center}
\begin{description}
  \item[複雜度:] $O(W\Sigma cnt_i)$
  \item[已知:] 第$i$個物品重量為$w_i$,價值$v_i$,有$cnt_i$個;\par
                背包總容量$W$
  \item[意義:] dp[前i個物品][重量] = 最高價值
  \item[maxn:] 物品數量
  \item[maxw:] 背包最大容量
\end{description}
\begin{lstlisting}
int W;
int w[maxn], v[maxn], cnt[maxn];
int dp[maxw];

memset(dp, 0, sizeof(dp));
for(int i=1; i<=n; i++)
  for(int j=W; j>=w[i]; j--)
    for(int k=1; k*w[i]<=j&&k<=cnt[i]; k++)
      dp[j] = max(dp[j],dp[j-k*w[i]]+k*v[i]);
\end{lstlisting}

\begin{center}
  \underline{  \bf\centering{混合背包(0-1/完全/多重)}  }
\end{center}
\begin{description}
  \item[複雜度:] $O(W\Sigma cnt_i)$
  \item[已知:] 第$i$個物品重量為$w_i$,價值$v_i$,有$cnt_i$個;\par
                背包總容量$W$
  \item[意義:] dp[前i個物品][重量] = 最高價值
  \item[maxn:] 物品數量
  \item[maxw:] 背包最大容量
  \item[$cnt_i=0$] 代表無限 
\end{description}
\begin{lstlisting}
int W;
int w[maxn], v[maxn], cnt[maxn];
int dp[maxw];

memset(dp, 0, sizeof(dp));
for(int i=1; i<=n; i++) {
  if(cnt[i]) {
    for(int j=W; j>=w[i]; j--) {
      for(int k=1;k*w[i]<=j&&k<=cnt[i];k++) {
        dp[j]=max(dp[j],dp[j-k*w[i]]+k*v[i]);
      }
    }
  } else {
    for(int j=w[i]; j<=W; j++) {
      dp[j] = max(dp[j], dp[j-w[i]] + v[i]);
    }
  }
}
\end{lstlisting}

\begin{center}
  \underline{  \bf\centering{二維費用背包}  }
\end{center}
\begin{description}
  \item[複雜度:] $O(NCT)$
  \item[已知:] 第$k$個任務需要花費$c_k$元,耗時$t_k$分鐘;\par
               總經費$C$,總耗時$T$
  \item[意義:] dp[前k個任務][花費][耗時] = 最多任務數
  \item[maxc:] 最大花費
  \item[maxt:] 最大耗時
\end{description}
\begin{lstlisting}
int C, T;
int c[maxn], t[maxn];
int dp[maxc][maxt];

memset(dp, 0, sizeof(dp));
for(int k=1; k<=n; k++)
  for(int i=C; i>=c[k]; i--)
    for(int j=T; j>=t[k]; j--)
      dp[i][j] = max(
        dp[i][j], dp[i-c[k]][j-t[k]] + 1);
\end{lstlisting}

\begin{center}
  \underline{  \bf\centering{分組背包}  }
\end{center}
\begin{description}
  \item[複雜度:] $O(W\Sigma M)$
  \item[已知:] 第$i$組第$j$個物品重量為$w_{ij}$,價值$v_{ij}$;\par
               背包總容量$W$;每組只能取一個
  \item[意義:] dp[前i組物品][重量] = 最高價值
  \item[maxn:] 物品組數
  \item[maxm:] 每組物品數
  \item[maxw:] 背包最大容量
\end{description}
\begin{lstlisting}
int W;
int dp[maxw];
vector<vector<int>> w, v;

memset(dp, 0, sizeof(dp));
for(int i=0; i<n; i++)
  for(int j=W; j>=0; j--)
    for(int k=0; k<w[i].size(); k++)
      if(j >= w[i][k])
        dp[j] = max(
          dp[j], dp[j-w[i][k]] + v[i][k]);
\end{lstlisting}

\begin{center}
  \underline{  \bf\centering{依賴背包}  }
\end{center}
\begin{description}
  \item[已知:] 第$j$個物品在第$i$個物品沒選的情況下不能選
  \item[做法:] 樹DP,有爸爸才有小孩。轉化為分組背包。
  \item[意義:] dp[選物品i為根][重量] = 最高價值
  \item[過程:] 對所有$u \rightarrow v$,dfs 計算完$v$後更新$u$
\end{description}

\begin{center}
  \underline{  \bf\centering{背包變化}  }
\end{center}

\begin{description}
  \item[1.] 求最大價值的方法總數 cnt
\end{description}
\begin{lstlisting}
for(int i=1; i<=n; i++) {
  for(int j=W; j>=w[i]; j--) {
    if(dp[j] < dp[j-w[i]]+v[i]) {
      dp[j] = dp[j-w[i]] + v[i];
      cnt[j] = cnt[j-w[i]];
    } else if(dp[j] == dp[j-w[i]]+v[i]) {
      cnt[j] += cnt[j-w[i]];
    }
  }
}
\end{lstlisting}

\begin{description}
  \item[2.] 求最大價值的一組方案pick
\end{description}
\begin{lstlisting}
memset(pick, 0, sizeof(pick));
for(int i=1; i<=n; i++) {
  for(int j=W; j>=w[i]; j--) {
    if(dp[i][j] < dp[i-1][j-w[i]]+v[i]) {
      dp[i][j] = dp[i-1][j-w[i]] + v[i];
      pick[i] = 1;
    } else {
      pick[i] = 0;
    }
  }
}
\end{lstlisting}

\begin{description}
  \item[3.] 求最大價值的字典序最小的一組方案pick
\end{description}
\begin{lstlisting}
// reverse(item), 要把物品順序倒過來
memset(pick, 0, sizeof(pick));
for(int i=1; i<=n; i++) {
  for(int j=W; j>=w[i]; j--) {
    if(dp[i][j] <= dp[i-1][j-w[i]]+v[i]) {
      dp[i][j] = dp[i-1][j-w[i]] + v[i];
      pick[i] = 1;
    } else {
      pick[i] = 0;
    }
  }
}
\end{lstlisting} \columnbreak
    % \subsection{以價值為主的背包}
    %     \lstinputlisting{DP/以價值為主的背包.cpp}
    \subsection{Range DP}
        \lstinputlisting{DP/RangeDP.cpp}
    \subsection{Deque最大差距}
        \lstinputlisting{DP/Deque最大差距.cpp} \columnbreak
    \subsection{string DP}
        Edit distance $S_1$ 最少需要經過幾次增、刪或換字變成 $S_2$
\begin{tiny}
    \[
        dp[i][j] = 
        \begin{cases}
            \begin{array}{cl}
                i+1, & \mbox{if} \ j=-1 \\
                j+1, & \mbox{if} \ i=-1 \\
                dp[i-1][j-1], & \mbox{if} \ S_1[i] = S_2[j] \\
                \min\left \{
                    \begin{array}{c}
                        dp[i][j-1] \\ dp[i-1][j] \\ dp[i-1][j-1]
                    \end{array}\right
                \}+1, & \mbox{if} \ S_1[i] \neq S_2[j]
            \end{array}
        \end{cases}
    \]
\end{tiny}

{\raggedright
    Longest Palindromic Subsequence \par
}
\begin{tiny}
    \[
        dp[l][r] =
        \begin{cases}
            \begin{array}{crc}
                1 & \mbox{if} & l = r \\
                dp[l+1][r-1] & \mbox{if} & S[l] = S[r] \\
                \max\{dp[l+1][r], dp[l][r-1]\} & \mbox{if} & S[l] \neq S[r] \\
            \end{array}
        \end{cases}
    \]
\end{tiny}
    \subsection{Barcode}
        \lstinputlisting{DP/BarCode.cpp}
    \subsection{LCS和LIS}
        \lstinputlisting{DP/LCSLIS.cpp} \columnbreak
    \subsection{樹DP有幾個path長度為k}
        \lstinputlisting{DP/TreeDP_有幾個path長度為k_無註解.cpp}
    \subsection{抽屜}
        \lstinputlisting{DP/抽屜_無註解.cpp} \columnbreak
    \subsection{TreeDP reroot}
        \lstinputlisting{DP/TreeDP_reroot_無註解.cpp} \columnbreak
    \subsection{WeightedLIS}
        \lstinputlisting{DP/WeightedLIS_無註解.cpp}
\clearpage

% \section{DP List}                   % 改到外面用 1 column
%     \lstinputlisting{DP/dplist.cpp}
    

% \section{Section2}
%     \subsection{thm}
%         \begin{itemize}

\item 中文測試 

\item $\sum \limits_{i=1}^n i^2 = \frac{n(n+1)(2n+1)}{6}$

\item $ \binom{x}{y} = \frac{x!}{y!(x-y)!}$

\item $ {\int_0^\infty \mathrm{e}^{-x}\,\mathrm{d}x}$

\end{itemize}

\begin{bmatrix}a & b \\c & d \end{bmatrix}


\end{document}
