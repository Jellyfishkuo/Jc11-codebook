\begin{lstlisting}

/*
    ax = b (mod m)
    此式若同除 gcd(a, m)
    則
    (a / gcd(a, m))x = (b / gcd(a, m)) (mod (m / gcd(a, m)))
    x有唯一解
    使用ex_gcd(a / gcd(a, m), m / gcd(a, m))可以得到x_0
    此時x的通解為
    x = x_0 + (m / gcd(a, m)) * t
    
    通解怎麼來的?
    已經得到特解 ax_0 = b (mod m)
    => ax_0 = b + k_0 m
    假設還有一個x1滿足
    ax_1 = b (mod m)
    => ax_1 = b + k_1 m
    兩式相減
    a(x_1 - x_0) = (k_1 - k_0) m
    可以觀察到
    a(x_1 - x_0) = 0 (mod m)
    兩邊同除gcd(a, m)
    => a / gcd(a, m) (x_1 - x_0) = 0 (mod (m / gcd(a, m)) )
    因為 (a / gcd(a, m)) 與 (m / gcd(a, m))一定互質
    所以只有可能
    x_1 - x_0 = 0 (mod (m / gcd(a, m)) ) 才能使前式成立
    至此就能夠得到通解
    x_1 = x_0 + t * (m / gcd(a, m)), t 是 正整數
    
    利用ex_gcd得到的通解可以得到k'_1
    此時可以帶回去前面的等式
    k'_1 * P - r_1 = r
    
    k'_1用算得拿到
    r_1已知 (題目有給)
    則P我們可以用枚舉的
*/

/*
    解ax + by = gcd(a, b) [貝祖等式]
    因為gcd(a, b) = gcd(b, a mod b)

    邊界條件:
        b = 0時
        ax + by = a * 1 + 0 * 0 = a = gcd
    然後
    將gcd(a, b) = gcd(b, a mod b)帶入ax + by = gcd(a, b)
    => bx' + (a mod b) y' = gcd(b, a mod b)
    根據除法定理
    a = b * 下高斯[a / b] + (a mod b)
    移項
    (a mod b) = a - b * 下高斯[a / b]
    帶入 bx' + (a mod b) y' = gcd(b, a mod b)
    => bx' + (a - b * 下高斯[a / b]) y' = gcd(a, b)
    整理
    => ay' + b(x' - 下高斯[a / b] y') = gcd(a, b)
    對應回ax + by = gcd(a, b)
    得到
    x = y', y = x' - 下高斯[a / b] * y'
*/

long long ex_gcd(long long a, long long b, long long& x, long long& y)
{
    if (b == 0)
    {
        x = 1;
        y = 0;
        return a;
    }

    long long gcd = ex_gcd(b, a % b, x, y);
    long long temp = x;
    x = y;
    y = temp - a / b * y;

    return gcd;
}

// 對應   ax = b (mod m)
// 又對回 P * k'_1 = 2r_1 (mod Q)
void solve(long long a, long long b, long long m)
{
    long long x, y;
    long long gcd = ex_gcd(a, m, x, y);

    // ax = b (mod m)有解的條件是 gcd(a, m) | b
    if (b % gcd == 0)
    {
        // ex_gcd得到 ax + my = gcd(a, m)的基本解
        // 兩邊同乘 b / gcd(a, m)
        // 得到 a (x * (b / gcd(a, m))) + m (y * (b / gcd(a, m))) = b
        // 在mod m的情況下
        // 得到   a (x * (b / gcd(a, m))) = b (mod m)
        // 對應回 ax_0 = b (mod m)
        // 所以取 x_0 = x * b / gcd(a, m)  [前兩個式子直接對應出 x_0 = x * b / gcd]
        long long x_0 = x * b / gcd; // 特解

        // 由於當前知道 ax_0 = b (mod m)
        // 通解是: x = x_0 + t * (m / gcd(a, m))
        // 為了讓x_0是通解裡面最小的正整數
        // 要 x_0 去 mod (m / gcd(a, m))
        // 但是因為現在x_0可能是負的
        // 所以mod (m / gcd(a, m))的時候要 加上 (m / gcd(a, m))
        x_0 = (x_0 % (m / gcd) + (m / gcd)) % (m / gcd);

        // 由於之前 r = k'_1 * P - r_1
        // 且求得k'_1 = x_0 + t * (m / gcd(a, m))
        // 代入可得
        // r = P( x_0 + t * (m / gcd(a, m)) ) - r_1
        // 對應回來 a = P, b = 2 * r_1
        // 則 r = a (x_0 + t * (m / gcd(a, m))) - b / 2
        // => r = ax_0 + t * (am / gcd(a, m)) - b / 2
        // => r = ax_0 + t * lcm(a, m) - b / 2
        // 所以下方利用累加lcm做到枚舉t的效果
        long long r = a * x_0 - b / 2;
        long long lcm = a / gcd * m;
        for (; r < N; r += lcm)
        {
            if (r >= 0 && (r * r) % N == X)
                res.insert(r);
        }
    }
}
\end{lstlisting}
