\begin{itemize}
    \item Tp 為 Point 裡 x 和 y 的型態
    \item struct Point 需要加入並另外計算的 variables: \\
        1. ang, 該點與基準點的atan2值 \\
        2. d2, 該點與基準點的$(距離)^2$
    \item 注意計算 d2 的型態範圍限制
\end{itemize}

\begin{center}
\underbar{Template}
\begin{lstlisting}
using DBL = double;
using Tp = long long;             // 存點的型態
const DBL eps = 1e-9;
const Tp inf = 1e9;               // 座標極大值
struct Vector;
using Point = Vector;
using Polygon = vector<Point>;
Vector operator-(Vector, Vector);
Tp cross(Vector, Vector);
int dcmp(DBL, DBL);
\end{lstlisting}
\underbar{Convex Hull}
\begin{lstlisting}
Polygon convex_hull(Point* p, int n) {
  auto rmv = [](Point a, Point b, Point c) {
    return cross(b-a, c-b) <= 0;  // 非浮點數
    return dcmp(cross(b-a, c-b)) <= 0;
  };

  // 選最下裡最左的當基準點,可在輸入時計算
  Tp lx = inf, ly = inf;
  for(int i=0; i<n; i++) {
    if(p[i].y<ly || (p[i].y==ly&&p[i].x<lx)){
      lx = p[i].x, ly = p[i].y;
    } 
  }

  for(int i=0; i<n; i++) {
    p[i].ang=atan2(p[i].y-ly,p[i].x-lx);
    p[i].d2 = (p[i].x-lx)*(p[i].x-lx) +
              (p[i].y-ly)*(p[i].y-ly);
  }
  sort(p, p+n, [&](Point& a, Point& b) {
    if(dcmp(a.ang, b.ang))
      return a.ang < b.ang;
    return a.d2 < b.d2;
  });

  int m = 1;     // stack size
  Point st[n] = {p[n] = p[0]};
  for(int i=1; i<=n; i++) {
    for(;m>1&&rmv(st[m-2],st[m-1],p[i]);m--);
    st[m++] = p[i];
  }
  return Polygon(st, st+m-1);
}
\end{lstlisting}
\end{center}