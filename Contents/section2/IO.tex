\begin{itemize}
\item 高維陣列先後順序
int arr[10][10000]; 優於 int arr[10000][10];

\item ++ , -- 放前面
for(int i=0; i<=n; i++)   //slower
for(int i=0; i<=n; ++i)   //faster

\item Pragma GCC Optimize
程式碼加上下面這行。
注意 : 如果程式碼里有遞迴的話,用O2會比O3快,因為O3優化會縮小堆疊空間。
#pragma GCC optimize("Ofast,fast-math,unroll-loops")

\item No Long Long
反正 long long 比 int慢很多。

\item 使用位元運算
乘以或除以2的冪次可以用 << 和 >>
判斷奇數偶數可以用 & 如果 x&1 == 1 就表示 x 是奇數
判斷不等於可以把 a != b 改成 a ^ b

\item 圖論存圖用鏈式前向星
就很快。不過聽說如果數據非常大的話(v=1e6,edge=4e6之類的),vector反而會比較快

\item 減少記憶體使用量
#include<bits/stdc++.h> → #include<stdio.h>
這檔案很大,你 include 它就要編譯它,會用很多記憶體。
using namespace std; → 自己寫std裡面的東西 , 或是用到的時候才在前面加上std::
\end{itemize}
